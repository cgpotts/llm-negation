\documentclass[11pt]{article}

\usepackage[utf8]{inputenc}
\usepackage[T1, OT2,OT1]{fontenc}
\usepackage{textcomp}
\usepackage[scale={0.81,0.95}, centering, includeheadfoot]{geometry}
\usepackage[charter]{mathdesign}
\usepackage{natbib}
\usepackage{amsmath}
\usepackage{xcolor}
\definecolor{black}{rgb}{0,0,0}
\usepackage[colorlinks, linkcolor=black, urlcolor=blue, citecolor=black]{hyperref}
\usepackage{graphicx}
\usepackage{fancyhdr}
\usepackage{booktabs}
\usepackage{inconsolata}

\newcommand{\promptExample}[1]{
    {\centering
    \textbf{Prompt example}\\
    \framebox{
      \begin{minipage}[c]{1\textwidth}
        #1
    \end{minipage}}}}

\newcommand{\mydblnewline}{{\color{gray}\\{\textbackslash n\textbackslash n\\}}}
\newcommand{\mynewline}{{\color{gray}{\textbackslash n\\}}}
\newcommand{\word}[1]{\emph{#1}}

%=====================================================================
%=====================================================================

\begin{document}

\pagestyle{fancy}
\lhead{}
\rhead{}
\lfoot{}
\cfoot{\thepage}
\rfoot{}
\renewcommand{\headrulewidth}{0pt}

\begin{center}
  {\Large\textbf{Does GPT-3 really understand negation?}} \\[1ex]
  Christopher Potts \\
  (with input from Selena She, Atticus Geiger, and
  Omar Khattab; all conclusions, and any mistakes, are my own)

\end{center}

%=====================================================================
%=====================================================================

\section{Background}

Sam Bowman wrote a blog post that included this claim:
%
\begin{quote}
  It's still easy to find extremely strong negative claims about these
  models from serious researchers, such as that LLMs don't have the
  capacity to handle negation (i.e., that they won’t be appropriately
  sensitive to it even in typical cases that look like language model
  training data). Ten minutes of interaction with the OpenAI API
  sandbox should be sufficient to refute this, but the academic
  conversation often anchors primarily on older systems or on smaller
  systems that are practical to run on academic clusters.

  \url{https://wp.nyu.edu/arg/why-ai-safety/} (October 6, 2022;
  updated October 8, 2022)
\end{quote}
%
On my reading, this is a claim that the best OpenAI models as of this
writing make correct predictions about simple cases involving
negation, and that this is self-evident from basic interactions with
these models.

If this is true, then it is a major discovery and a significant sign
of progress. If it is incorrect, then it raises real concerns for me
as someone who cares about this topic. Bowman is very influential, and
statements like this can shape people's attitudes about what it is
worthwhile to work on.

No evidence is provided in the above post. Free-form interaction with
the API is very unlikely to lead to reliable evidence, since it is
unlikely to involve systematic exploration of a wide range of cases
with careful, predefined criteria for success. In general, free-form
interaction leaves one susceptible to confirmation bias.

In this document, I report on my own investigations. The high-level
summary is that in a balanced test set of 200 diverse sentences
involving basic interactions of negation with lexical entailment
relations, the best I was able to do was 62\% accuracy (chance is
50\%).

It may be that there are prompting strategies that will yield better
results, but I do not think it is trivial to find them, and so I think
this actually challenges Bowman's characterization above. It may be
that Bowman agrees with this fundamental point, since the following
thread asserts that the desired behavior can be elicited with very
careful methods (which are not disclosed in the thread as of this
writing):
\\
\url{https://twitter.com/sleepinyourhat/status/1578878667207294977}


\newpage

\section{Dataset structure}

This dataset was created by Selena She and Atticus Geiger this past
summer. It extends \href{https://github.com/atticusg/MoNLI}{the MoNLI
  dataset} with lots of different kinds of negated example.
%
\begin{center}
  \footnotesize
  \begin{tabular}{@{} l p{0.33\textwidth} c p{0.33\textwidth} @{}}
    \toprule
    split &sentence1 &  gold\_label & sentence2 \\
    \midrule
    one scoping &       The guests do not have food. &  entailment &  The guests do not have rice. \\
    one scoping &       A child is not tossing blocks &     neutral & A child is not tossing anything \\\midrule
    both not scoping &  The not allergic person did like cranberries but prefers not to eat vegetables. &  entailment &  The not allergic person did like fruit but prefers not to eat vegetables. \\
    both not scoping &  A man who is not here is holding something, but not in his hands. &     neutral & A man who is not here is holding stamps, but not in his hands. \\\midrule
    one not scoping &   The skier noticed a large elm branch dragging behind him but did not move it. &  entailment &  The skier noticed a large tree branch dragging behind him but did not move it. \\
    one not scoping &   the boy plays an instrument, but he does not want to play anymore. &     neutral & the boy plays a fiddle, but he does not want to play anymore. \\\midrule
    one scoping, one not & People who are not wealthy do not have any computers &  entailment & People who are not wealthy do not have any laptops \\
    one scoping, one not & The woman who did not like clothes did not like meatloaf. &  neutral & The woman who did not like clothes did not like meat. \\
    \bottomrule
  \end{tabular}
\end{center}


\section{Methods}

\begin{enumerate}
\item Engine: \texttt{text-davinci-002}
\item The experimental runs given below were conducted on October 9, 2022.
\item Temperature: 0.0
\item Per-split distribution:

\begin{tabular}{lrr}
\toprule
 &  entailment &  neutral \\
\midrule
both not scoping     &          25 &       25 \\
one not scoping      &          25 &       25 \\
one scoping          &          25 &       25 \\
one scoping, one not &          25 &       25 \\
\bottomrule
\end{tabular}

\item Label inference: If the response contains ``yes'' (case-insensitive),
  then we infer the label \texttt{entailment}, else we infer
  \texttt{neutral}.

\item Across experiments, the only thing that varies is the nature of
  the prompt function.

\item Below, I've indicated all actual newlines with
  {\color{gray}\textbackslash n}. The newlines in
  the latex formatting are just to make them intuitive to read.

\end{enumerate}

\newpage

\section{Experiments}

\subsection{Conditional Question Prompt}

\promptExample{Is it true that if we didn't eat pizza, then we didn't eat food?}

\begin{center}
\begin{tabular}{lrrr}
\toprule
{} &  Incorrect &  Correct &  Accuracy \\
\midrule
both not scoping     &         22 &       28 &      0.56 \\
one not scoping      &         23 &       27 &      0.54 \\
one scoping          &         25 &       25 &      0.50 \\
one scoping, one not &         25 &       25 &      0.50 \\
All                  &         95 &      105 &      0.53 \\
\bottomrule
\end{tabular}
\end{center}

%%%%%%%%%%%%%%%%%%%%%%%%%%%%%%%%%%%%%%%%%%%%%%%%%%%%%%%%%%%%%%%%%%%%%%
\subsection{Few-Shot Conditional Question Prompt}

\promptExample{Q1: Is it true that if the men were outside trying to keep their voices down so as not to waken any woman indoors, then the men were outside trying to keep their voices down so as not to waken any bachelorette indoors?\mynewline
A1: Yes\mynewline
\mynewline
Q2: Is it true that if the person did not like any fruit, then the person did not like any cranberries?\mynewline
A2: Yes\mynewline
\mynewline
Q3: Is it true that if the woman is not wearing earrings, then the woman is not wearing jewelry?\mynewline
A3: Maybe\mynewline
\mynewline
Q4: Is it true that if the diver has not seen any tuna on his dive, then the diver has not seen any fish on his dive?\mynewline
A4: Maybe\mynewline
\mynewline
Q: Is it true that if we didn't eat pizza, then we didn't eat food?\mynewline
A:}

\begin{center}
\begin{tabular}{lrrr}
\toprule
{} &  Incorrect &  Correct &  Accuracy \\
\midrule
both not scoping     &         15 &       35 &      0.70 \\
one not scoping      &         19 &       31 &      0.62 \\
one scoping          &         26 &       24 &      0.48 \\
one scoping, one not &         26 &       24 &      0.48 \\
All                  &         86 &      114 &      0.57 \\
\bottomrule
\end{tabular}
\end{center}

\newpage

%%%%%%%%%%%%%%%%%%%%%%%%%%%%%%%%%%%%%%%%%%%%%%%%%%%%%%%%%%%%%%%%%%%%%%
\subsection{Hypothesis Question Prompt}

\promptExample{Assume that we didn't eat pizza. Is it then definitely true that we didn't eat food? Answer Yes or No.}

\begin{center}
\begin{tabular}{lrrr}
\toprule
{} &  Incorrect &  Correct &  Accuracy \\
\midrule
both not scoping     &         20 &       30 &      0.60 \\
one not scoping      &         22 &       28 &      0.56 \\
one scoping          &         21 &       29 &      0.58 \\
one scoping, one not &         21 &       29 &      0.58 \\
All                  &         84 &      116 &      0.58 \\
\bottomrule
\end{tabular}
\end{center}

%%%%%%%%%%%%%%%%%%%%%%%%%%%%%%%%%%%%%%%%%%%%%%%%%%%%%%%%%%%%%%%%%%%%%%
\subsection{Few-Shot Hypothesis Question Prompt}

\promptExample{Q1: Assume that the two people are not fishing from a boat. Is it then definitely true that the two people are not fishing from a gondola? Answer Yes or No.\mynewline
A1: Yes\mynewline
\mynewline
Q2: Assume that the guests do not have food. Is it then definitely true that the guests do not have rice? Answer Yes or No.\mynewline
A2: Yes\mynewline
\mynewline
Q3: Assume that the diver has not seen any tuna on his dive. Is it then definitely true that the diver has not seen any fish on his dive? Answer Yes or No.\mynewline
A3: No\mynewline
\mynewline
Q4: Assume that two people are not fishing from a skiff. Is it then definitely true that two people are not fishing from a boat? Answer Yes or No.\mynewline
A4: No\mynewline
\mynewline
Q: Assume that we didn't eat pizza. Is it then definitely true that we didn't eat food? Answer Yes or No.\mynewline
A:}

\begin{center}
\begin{tabular}{lrrr}
\toprule
{} &  Incorrect &  Correct &  Accuracy \\
\midrule
both not scoping     &         10 &       40 &      0.80 \\
one not scoping      &         19 &       31 &      0.62 \\
one scoping          &         21 &       29 &      0.58 \\
one scoping, one not &         25 &       25 &      0.50 \\
All                  &         75 &      125 &      0.62 \\
\bottomrule
\end{tabular}
\end{center}

\newpage

%%%%%%%%%%%%%%%%%%%%%%%%%%%%%%%%%%%%%%%%%%%%%%%%%%%%%%%%%%%%%%%%%%%%%%
\subsection{Conditional Truth Evaluation Prompt}

\promptExample{If we didn't eat pizza, then we didn't eat food. Is this true?}

\begin{center}
\begin{tabular}{lrrr}
\toprule
{} &  Incorrect &  Correct &  Accuracy \\
\midrule
both not scoping     &         21 &       29 &      0.58 \\
one not scoping      &         20 &       30 &      0.60 \\
one scoping          &         25 &       25 &      0.50 \\
one scoping, one not &         25 &       25 &      0.50 \\
All                  &         91 &      109 &      0.55 \\
\bottomrule
\end{tabular}
\end{center}

%%%%%%%%%%%%%%%%%%%%%%%%%%%%%%%%%%%%%%%%%%%%%%%%%%%%%%%%%%%%%%%%%%%%%%
\subsection{Few-Shot Conditional Truth Evaluation Prompt}

\promptExample{C1: If the people are not playing instruments, then the people are not playing flutes. Is this true?\mynewline
A1: Yes\mynewline
\mynewline
C2: If the guests do not have food, then the guests do not have rice. Is this true?\mynewline
A2: Yes\mynewline
\mynewline
C3: If the people are not playing tambourines, then the people are not playing instruments. Is this true?\mynewline
A3: Maybe\mynewline
\mynewline
C4: If there is not a single painter walking in the city, then there is not a single person walking in the city. Is this true?\mynewline
A4: Maybe\mynewline
\mynewline
C:If we didn't eat pizza, then we didn't eat food. Is this true?\mynewline
A:}

\begin{center}
\begin{tabular}{lrrr}
\toprule
{} &  Incorrect &  Correct &  Accuracy \\
\midrule
both not scoping     &          8 &       42 &      0.84 \\
one not scoping      &         13 &       37 &      0.74 \\
one scoping          &         28 &       22 &      0.44 \\
one scoping, one not &         30 &       20 &      0.40 \\
All                  &         79 &      121 &      0.60 \\
\bottomrule
\end{tabular}
\end{center}

\newpage

%%%%%%%%%%%%%%%%%%%%%%%%%%%%%%%%%%%%%%%%%%%%%%%%%%%%%%%%%%%%%%%%%%%%%%
\subsection{Brown et al.-style Prompt}

\promptExample{C: We didn't eat pizza\mynewline
Q: We didn't eat food. Yes, No, or Maybe?}

\begin{center}
\begin{tabular}{lrrr}
\toprule
{} &  Incorrect &  Correct &  Accuracy \\
\midrule
both not scoping     &         22 &       28 &      0.56 \\
one not scoping      &         23 &       27 &      0.54 \\
one scoping          &         25 &       25 &      0.50 \\
one scoping, one not &         25 &       25 &      0.50 \\
All                  &         95 &      105 &      0.53 \\
\bottomrule
\end{tabular}
\end{center}

%%%%%%%%%%%%%%%%%%%%%%%%%%%%%%%%%%%%%%%%%%%%%%%%%%%%%%%%%%%%%%%%%%%%%%
\subsection{Few-Shot Brown et al.-style Prompt}

\promptExample{C1: The man is not listening to music.\mynewline
Q1: The man is not listening to reggae. Yes, No, or Maybe?\mynewline
A2: Yes\mynewline
\mynewline
C2: The woman is not wearing jewelry\mynewline
Q2: The woman is not wearing necklaces. Yes, No, or Maybe?\mynewline
A3: Yes\mynewline
\mynewline
C3: A woman did not like soup.\mynewline
Q3: A woman did not like food. Yes, No, or Maybe?\mynewline
A4: Maybe\mynewline
\mynewline
C4: A child is not tossing blocks\mynewline
Q4: A child is not tossing anything. Yes, No, or Maybe?\mynewline
A5: Maybe\mynewline
\mynewline
C: We didn't eat pizza\mynewline
Q: We didn't eat food. Yes, No, or Maybe?\mynewline
A:}

\begin{center}
\begin{tabular}{lrrr}
\toprule
{} &  Incorrect &  Correct &  Accuracy \\
\midrule
both not scoping     &         12 &       38 &      0.76 \\
one not scoping      &         16 &       34 &      0.68 \\
one scoping          &         31 &       19 &      0.38 \\
one scoping, one not &         33 &       17 &      0.34 \\
All                  &         92 &      108 &      0.54 \\
\bottomrule
\end{tabular}
\end{center}

\newpage

%%%%%%%%%%%%%%%%%%%%%%%%%%%%%%%%%%%%%%%%%%%%%%%%%%%%%%%%%%%%%%%%%%%%%%
\subsection{Structured Prompt}

\promptExample{P: We didn't eat pizza\mynewline
H: We didn't eat food\mynewline
L:}

\begin{center}
\begin{tabular}{lrrr}
\toprule
{} &  Incorrect &  Correct &  Accuracy \\
\midrule
both not scoping     &         25 &       25 &      0.50 \\
one not scoping      &         25 &       25 &      0.50 \\
one scoping          &         25 &       25 &      0.50 \\
one scoping, one not &         25 &       25 &      0.50 \\
All                  &        100 &      100 &      0.50 \\
\bottomrule
\end{tabular}
\end{center}

%%%%%%%%%%%%%%%%%%%%%%%%%%%%%%%%%%%%%%%%%%%%%%%%%%%%%%%%%%%%%%%%%%%%%%
\subsection{Few-Shot Structured Prompt}

\promptExample{P1: A woman did not like food.\mynewline
H1: A woman did not like burritos.\mynewline
L1: entailment\mynewline
\mynewline
P2: There is not a single person walking in the city.\mynewline
H2: There is not a single official walking in the city.\mynewline
L2: entailment\mynewline
\mynewline
P3: People do not have any computers\mynewline
H3: People do not have any machines\mynewline
L3: neutral\mynewline
\mynewline
P4: There is not a tugboat nearby.\mynewline
H4: There is not a boat nearby.\mynewline
L4: neutral \mynewline
\mynewline
P: We didn't eat pizza\mynewline
H: We didn't eat food\mynewline
L:}

\begin{center}
\begin{tabular}{lrrr}
\toprule
{} &  Incorrect &  Correct &  Accuracy \\
\midrule
both not scoping     &         25 &       25 &      0.50 \\
one not scoping      &         25 &       25 &      0.50 \\
one scoping          &         25 &       25 &      0.50 \\
one scoping, one not &         25 &       25 &      0.50 \\
All                  &        100 &      100 &      0.50 \\
\bottomrule
\end{tabular}
\end{center}

\newpage

%%%%%%%%%%%%%%%%%%%%%%%%%%%%%%%%%%%%%%%%%%%%%%%%%%%%%%%%%%%%%%%%%%%%%%
\subsection{Reasoning Prompt}

\promptExample{Logical and commonsense reasoning exam.\mynewline
\mynewline
Explain your reasoning in detail, then answer with Yes or No. Your answers should follow this 4-line format:\mynewline
\mynewline
Premise: <a tricky logical statement about the world>.\mynewline
Question: <question requiring logical deduction>.\mynewline
Reasoning: <an explanation of what you understand about the possible scenarios>.\mynewline
Answer: <Yes or No>.\mynewline
\mynewline
Premise: we didn't eat pizza \mynewline
Question: Can we logically conclude for sure that we didn't eat food? \mynewline
Reasoning: Let's think logically step by step. The premise basically tells us that}

\begin{center}
\begin{tabular}{lrrr}
\toprule
{} &  Incorrect &  Correct &  Accuracy \\
\midrule
both not scoping     &         19 &       31 &      0.62 \\
one not scoping      &         20 &       30 &      0.60 \\
one scoping          &         21 &       29 &      0.58 \\
one scoping, one not &         19 &       31 &      0.62 \\
All                  &         79 &      121 &      0.60 \\
\bottomrule
\end{tabular}
\end{center}


\subsection{Reasoning Prompt Error Analysis}

The reasoning prompt responses might provide some insights into how
systematic the model's behavior actually is. Some observations:
%
\begin{enumerate}
\item For premise \word{The man is not allergic to any snakes} and
  hypothesis \word{The man is not allergic to any pythons} (an
  entailment
  case), the model continues from ``The premise basically tells us
  that'' with

  \texttt{the man has no allergies to snakes in general. That doesn't
    mean that he can't be allergic to a specific type of snake, such
    as a python. So the answer is No, we cannot logically conclude for
    sure that the man is not allergic to any pythons.}

  A charitable interpretation is that it has interpreted the \word{any}
  forms as generics. This explains a lot of the mistakes it makes
  in these simple negation cases.

\item Relatedly, some things labeled mistakes seems defensible. For
  example, for
  premise  \word{The woman liked corn, not exercise.} and hypothesis
  \word{The woman liked food, not exercise.} (an entailment case), the
  model continues from ``The premise basically tells us that'' with

  \texttt{the woman liked corn, not exercise. We can't assume that she
    liked food just because she liked corn. She might just have a
    specific taste for corn. We can't logically conclude that she
    liked food, not exercise.\mynewline
    Answer: No}

  This can be read as reflecting how unstable reasoning is with
  generics.

\item Very often, the model seems to get lost as it reasons,
  leading to a jumble of reasoning steps that do not form a coherent
  argument. For example, for premise \word{The people not wearing blue
    are not playing instruments}
  and hypothesis \word{The people not wearing blue are not playing
    tambourines.} (an entailment case), the model continues from ``The
  premise basically tells us that'' with

  \texttt{there are two groups of people- those not wearing blue and those
  playing instruments. It's possible that there are people who are not
  wearing blue and are playing instruments, but we can't say for sure.
  It's also possible that there are people who are not wearing blue
  and are not playing instruments. We can't say for sure that the
  people not wearing blue are not playing tambourines, but it's more
  likely than the alternative.\mynewline
  Answer: No}

\item I was struck by the fact that the model caught one kind of case
  that shouldn't be in our dataset. For premise \word{The boy can find
    a turtle, just not now.} and hypothesis \word{The boy can find a
    reptile, just not now.}, we assign the label entailment. However,
  the model tends to give answers that convey neutral, with the
  reasoning that \word{just not now} has created a non-monotonic
  reasoning environment.

\item Sometimes it seems like the model is simply reasoning
  incorrectly about exactly what we care about.
  For example, for premise \word{A woman liked food, not exercise}
  and hypothesis \word{A woman liked pudding, not exercise},
  the model continues from ``The
  premise basically tells us that'' with

  \texttt{the woman liked food and not exercise. So, we can logically
    conclude that she liked pudding and not exercise.\mynewline
    Answer: Yes}


\end{enumerate}

%=====================================================================
%=====================================================================


\end{document}
